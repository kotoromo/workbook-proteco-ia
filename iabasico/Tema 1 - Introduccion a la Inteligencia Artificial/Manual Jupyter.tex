\documentclass{article}
\usepackage[utf8]{inputenc}
\usepackage[spanish]{babel}
\usepackage{hyperref}
\usepackage{graphicx}
\usepackage{subcaption}
\usepackage{float}
\usepackage[]{algorithm2e}

\graphicspath{ {./manual_img/} }
\hypersetup{
    colorlinks=true,
    linkcolor=black,
    filecolor=magenta,      
    urlcolor=cyan,
}

\begin{document}
    \tableofcontents
    \newpage

    \section{Instalaci\'{o}n de Python 3.6 con Anaconda}
        \subsection{Windows}
            La instalaci\'{o}n de Anaconda Navigator es muy sencilla en Windows:
            \begin{enumerate}
                \item Acceder a la \href{https://www.anaconda.com/download/}{p\'{a}gina oficial de descarga} de 
                    Anaconda.
                \item Descargar la versi\'{o}n de Anaconda para Python 3.6.
                \begin{figure}[H]
                    \centering
                    \includegraphics[width=\textwidth]{anaconda1}
                    \caption{Dar click en Download para descargar.}
                \end{figure}

                \item Ejecutar el instalador descargado.
                \begin{figure}[H]
                    \centering
                    \includegraphics[width=\textwidth]{anaconda2}
                    \caption{Dar doble click.}
                \end{figure}

                \begin{figure}[H]
                    \centering
                    \includegraphics[width=\textwidth]{anaconda3}
                    \caption{Dejar las cajas sin marcar y dar click en Install.}
                \end{figure}

                \item ¡Felicidades! Ya tienes Anaconda Instalado.
            \end{enumerate}

            \newpage

        \subsection{GNU/Linux}
            \begin{enumerate}
                \item Acceder a la \href{https://www.anaconda.com/download/}{p\'{a}gina oficial de descarga} de 
                    Anaconda.
                \item Descargar la versi\'{o}n de Anaconda para Python 3.6.

                \item Correr el script: \textit{bash ~/Downloads/Anaconda3-5.2.0-Linux-x86\_64.sh}

                \item Seguir las indicaciones del instalador.

                \item Al finalizar el instalador, reiniciar la sesi\'{o}n de bash/zshell/lo que sea para
                que los cambios hagan efecto.

                \item Para abrir Anaconda Navigator, ejecutar el siguiente comando: \textit{anaconda-navigator}
            \end{enumerate}
            \newpage
    \section{Instalaci\'{o}n de Python 2.7 con Anaconda}
        \subsection{Windows}
        
        El proceso es el mismo que con Python 3.6; la diferencia radica en que hay que seleccionar para descargar
        el instalador de Anaconda con Python 2.7.
        \subsection{GNU/Linux}
        El proceso es el mismo que con Python 3.6; la diferencia radica en que hay que seleccionar para descargar
        el instalador de Anaconda con Python 2.7.

    \section{Instalaci\'{o}n de Jupyter}
        \subsection{Windows}
            Abrir Anaconda Navigator y, una vez en el dashboard, seleccionar 'Install' en el recuadro que
            ofrece instalar Jupyter.
        \subsection{GNU/Linux}
            Abrir Anaconda Navigator y, una vez en el dashboard, seleccionar 'Install' en el recuadro que
            ofrece instalar Jupyter.
    
    \newpage
    \section{Manual de Markdown para dar formato a celdas en Jupyter}
        \begin{itemize}
            \item Para incluir \textbf{LaTeX}:\$[C\'{o}digo LaTeX aqu\'{i}]\$
            \item Para hacer negritas: **\textbf{[Texto en negritas]}** o \_\_\textbf{[Texto en negritas]}\_\_
            \item Para hacer italics: *\textit{[Texto en italics]}* o \_\textit{[Texto en italics]}\_ 
            \item Para hacer listas: 
            \begin{figure}[H]
                \begin{subfigure}{0.5\textwidth}
                    \includegraphics[width=0.9\linewidth]{lista1}
                    \caption{As\'{i} se crean las listas.}
                \end{subfigure}
                \begin{subfigure}{0.5\textwidth}
                    \includegraphics[width=0.9\linewidth]{lista2}
                    \caption{As\'{i} lucen tras ejecutar el c\'{o}digo Markdown.}
                \end{subfigure}
            \end{figure}

            \item As\'{i} se incluyen enlaces: [Google](https://www.google.com)

            \item As\'{i} se incluyen im\'{a}genes de Internet: ![alt text](enlace\_a\_imagen)
            \item Para hacer tablas:
            \begin{figure}[H]
                \begin{subfigure}{0.5\textwidth}
                    \includegraphics[width=0.9\linewidth]{tabla1}
                    \caption{As\'{i} se crean las tablas.}
                \end{subfigure}
                \begin{subfigure}{0.5\textwidth}
                    \includegraphics[width=0.9\linewidth]{tabla2}
                    \caption{As\'{i} lucen tras ejecutar el c\'{o}digo Markdown.}
                \end{subfigure}
            \end{figure}
        \end{itemize}
    \newpage
        \section{Manual de uso b\'{a}sico de Jupyter}
            \begin{itemize}
                \item Para crear un nuevo documento:
                \begin{figure}[H]
                    \centering
                    \includegraphics[width=\textwidth]{Jupyter1}
                    \caption{Seleccionar la opci\'{o}n de Python 3.}
                \end{figure}

                \item Para ejecutar c\'{o}digo de Python, escribir el c\'{o}digo en una celda
                vac\'{i}a y presionar la combinaci\'{o}n de teclas $CTRL+Enter$.

                \begin{figure}[H]
                    \begin{subfigure}{0.5\textwidth}
                        \includegraphics[width=0.9\linewidth]{Jupyter2}
                        \caption{Celda vac\'{i}a.}
                    \end{subfigure}
                    \begin{subfigure}{0.5\textwidth}
                        \includegraphics[width=0.9\linewidth]{Jupyter3}
                        \caption{Celda tras ejecutar el c\'{o}digo.}
                    \end{subfigure}
                \end{figure}

                \item Para insertar una nueva celda:
                \begin{figure}[H]
                    \begin{subfigure}{0.5\textwidth}
                        \includegraphics[width=0.9\linewidth]{Jupyter4}
                        \caption{Seleccionar $insert$ y luego cualquiera de las dos opciones.}
                    \end{subfigure}
                    \begin{subfigure}{0.5\textwidth}
                        \includegraphics[width=0.9\linewidth]{Jupyter5}
                        \caption{Insert below inserta la celda debajo de la celda actual. Insert above
                        inserta la nueva celda encima de la celda actual.}
                    \end{subfigure}
                \end{figure}

                \item Para eliminar una celda primero hay que seleccionar la celda a eliminar.
                \begin{figure}[H]
                    \begin{subfigure}{0.5\textwidth}
                        \includegraphics[width=0.9\linewidth]{Jupyter5}
                        \caption{Celda seleccionada. Se observa un color a la izquierda que indica
                        la selecci\'{o}n.}
                    \end{subfigure}
                    \begin{subfigure}{0.5\textwidth}
                        \includegraphics[width=0.9\linewidth]{Jupyter6}
                        \caption{Seleccionar \textit{Delete Cells}}
                    \end{subfigure}
                \end{figure}

                \newpage
                \item Para cambiar el tipo de celda a \textit{Markdown}:
                \begin{figure}[H]
                    \begin{subfigure}{0.5\textwidth}
                        \includegraphics[width=0.9\linewidth]{Jupyter5}
                        \caption{Seleccionar la celda deseada.}
                    \end{subfigure}
                    \begin{subfigure}{0.5\textwidth}
                        \includegraphics[width=0.9\linewidth]{Jupyter7}
                        \caption{Seleccionar \textit{Markdown}. Observe como cambia la celda.}
                    \end{subfigure}
                \end{figure}


            \end{itemize}

            \newpage

            \begin{algorithm}[H]
                \SetAlgoLined
                \KwResult{Write here the result }
                 initialization\;
                 \While{While condition}{
                  instructions\;
                  \eIf{condition}{
                   instructions1\;
                   instructions2\;
                   }{
                   instructions3\;
                  }
                 }
                 \caption{How to write algorithms}
            \end{algorithm}
\end{document}